\documentclass[a4paper,14pt]{article}
%\usepackage[utf8]{inputenc}

\usepackage{cmap} % поиск в PDf
\usepackage{mathtext} % русские буквы в формулах
\usepackage[T2A]{fontenc} %  кодировка
\usepackage[utf8]{inputenc} % кодировка исходного текста
\usepackage[english,russian]{babel} % локализация и переносы
\usepackage{soulutf8} 
\usepackage{hyperref} % Ссылки на текст, формулы

% Макрос для удобной записи углов----------------
\def\ang#1#2{$#1^\circ\,#2'$} 
% ang{grad}{min}
%------------------------------------------------

% Вставка рисунков ------------------------------
\usepackage{graphicx} % вставка рисунков
\graphicspath{ {./images/} }
% -----------------------------------------------
\usepackage{extsizes} % для 14-го шрифта
\usepackage{setspace}
\singlespacing % междустрочной интервал = 1
\setstretch{1.15} % scale = 1.15

\newcommand\indent[1][1cm]{\hspace*{#1}}

\usepackage{geometry} % Поля
\geometry{top=20mm}
\geometry{bottom=20mm}
\geometry{left=20mm}
\geometry{right=20mm}

\usepackage{fancyhdr} % колонтитулы
\pagestyle{fancy}
\renewcommand{\headrulewidth}{0mm} % толщина линей

\title{MSGGO-Lab2}
\author{Андрей Юрьевич Будо }
\date{March 2021}

\usepackage{multicol}

\usepackage{amsmath,amsfonts,amssymb,amsthm,mathtools}
\usepackage{icomma}

\usepackage{hyperref}
\usepackage[usenames,dvipnames,svgnames,indentle,rgb]{xcolor}
\hypersetup{
    unicode=true, %
    colorlinks=true,
    linkcolor=blue
}

\usepackage{bm}

\begin{document}
    
    \thispagestyle{empty}
    \begin{center}
        МИНИСТЕРСТВО ОБРАЗОВАНИЯ РЕСПУБЛИКИ БЕЛАРУСЬ\\
        Белорусский Национальный Технический Университет\\
        Факультет транспортных коммуникаций\\
        Кафедра "Геодезия и аэрокосмические геотехнологии"
    \end{center}
    \begin{center}
        \vspace{13ex}
        ОТЧЁТ\\
        по лабораторной работе №2\\
        <<Прямая и обратная геодезические задачи на эллипсоиде>>\\
        Вариант 3
    \end{center}
    \begin{flushright}
        \vspace{13ex}
        Выполнил: \textit{Вишняков Д.Н.,\\
        студент группы 11405118}\\
        \vspace{5ex}
        Проверил: \textit{Будо А.Ю.,\\
        ст.пр.каф."ГиАКГТ"}\\
    \end{flushright}
    \begin{center}
        \vfill
        Минск, 2021
    \end{center}

\newpage
\section{Общие сведения}
\indent Главные геодезические задачи. Главными называют прямую и обратную геодезические
задачи. Пример прямой задачи – известны широта и долгота пункта А, азимут на пункт В и
расстояние до него; найти широту и долготу пункта В. Обратная задача – известны широта идолгота пунктов А и В, найти расстояние между ними, прямой и обратный азимуты. Рассмотрим их решение на сфере, а затем на эллипсоиде вращения. Для многих картографических задач решения на сфере по точности являются приемлемыми. Однако в настоящее время системы спутникового позиционирования позволяют определять координаты точек с погрешностями от нескольких метров до первых сантиметров. Поэтому целесообразно иметь представление и о более точных способах решения главных геодезических задач.\\
\begin{figure}[h]
    \centering
    \hypertarget{table1}{\includegraphics[width = 90mm]{images/Ellipsoid.jpg}}
    \caption{Земной эллипсоид вращения}
    \label{fig:my_label}
\end{figure}

\indent В данной задаче рассматривается эллипсоид WGS84 со следующими параметрами:
\begin{itemize}
    \item a = 6 378 137 - экваториальная (большая) полуось
    \item b = 6 356 863,018773 - полярная (малая) полуось
    \item a = \textit{f} = $\frac{a-b}{a} = \frac{1}{298.3}$ - полярное сжатие
    \item $\frac{1}{a} = \frac{1}{f} = \frac{a}{a-b} = 298.257223563$ - обратное сжатие\\
\end{itemize}

\begin{table}[h]
    \centering
    \caption{Исходные данные}
    \begin{tabular}{ |c|c|c|c| }
    \hline
         \textbf{№ п/п} & \textbf{Город} & \textbf{Широта, с.ш. \bm{$\phi$}} & \textbf{Долгота, в.д. \bm{$\lambda$}} \\
         \hline
         12 & Брест & \ang{52}{06} & \ang{23}{42} \\
         \hline
         13 & Высокое & \ang{52}{06} & \ang{23}{24} \\
    \hline
    \end{tabular}
\end{table}

\indent Для решения прямой и обратной задач на эллипсоиде воспользуемся формулами со средними аргументами (предложены К.Ф. Гауссом)

\newpage
\section{Обратная геодезическая задача на эллипсоиде}
Чтобы при решении геодезических задач на эллипсоиде формулы имели более компнактный вид, введём ряд обозначений:

\begin{equation}
    \hypertarget{f_1}{f_1 = \frac{1}{M}}
\end{equation}

\begin{equation}
    \hypertarget{f_2}{f_2 = \frac{1}{N}}
\end{equation}

\begin{equation}
    \hypertarget{f_3}{f_3 = \frac{1}{24}}
\end{equation}

\begin{equation}
    \hypertarget{f_4}{f_4 = \frac{1 + \eta^2 - 9 \cdot \eta^2 \cdot t^2}{24 \cdot V^4}}
\end{equation}

\begin{equation}
    \hypertarget{f_5}{f_5 = \frac{1 - 2 \cdot \eta^2}{24}}
\end{equation}

\begin{equation}
    \hypertarget{f_6}{f_6 = \frac{\eta^2 \cdot (1 - t^2) }{8 \cdot V^4}}
\end{equation}

\begin{equation}
    \hypertarget{f_7}{f_7 = \frac{1 + \eta^2}{12}}
\end{equation}

\begin{equation}
    \hypertarget{f_8}{f_8 = \frac{3 + 8 \cdot \eta^2}{24 \cdot V^4}}
\end{equation}

\indent При решении задачи обозначим широту и долготу начальных точек (\hyperlink{table1}{Табл. 1}) как:

\begin{center}
    $P_1(\phi_1, \lambda_1)$ $P_2(\phi_2, \lambda_2)$\\
\end{center}

\indent Вычисляем разности широт и долгот, а также среднее значение широты:

\begin{equation}
    \Delta\phi = \phi_2 - \phi_1
\end{equation}

\begin{equation}
    \Delta\lambda = \lambda_2 - \lambda_1
\end{equation}

\begin{equation}
    \phi = \frac{\phi_1 + \phi_2}{2}
\end{equation}

\indent Полученные значения:

\begin{center}
    $\Delta\phi =0.000291^\circ$ \\
   $\Delta\lambda = -0.005236^\circ$ \\
   $\phi = 0.909462^\circ$\\
\end{center}

\indent Далее вычисляем радиусы кривизны первого вертикала и меридиана:

\begin{equation}
    N = \frac{a}{\sqrt{1 - e^2 \cdot (\sin{\phi})}}
\end{equation}

\begin{equation}
    M = \frac{a \cdot (1 - e^2)}{(1 - e^2 \cdot \sin^2{\phi})^\frac{3}{2}}
\end{equation}

где е - квадрат эксцентриситета:

\begin{equation}
    e^2 = 2 \cdot f - f^2
\end{equation}
а, \textit{f} - параметры эллипсоида;\\
а также вспомогательные величины:

\begin{equation}
    t = \tan{\phi}
\end{equation}

\begin{equation}
    \eta^2 = \frac{e^2}{1 - e^2} \cdot \cos^2{\phi}
\end{equation}

\begin{equation}
    V^2 = 1 + \eta^2
\end{equation}

\indent и функции \textit{$f_1$} \textit{$f_2$} по формулам (\hyperlink{f_1}{1}) - (\hyperlink{f_8}{8})

\indent Полученные значения:
\begin{center}
    $N = 6391474.662172 \textit{м}$\\
    $M = 6375267.622755 \textit{м}$\\
    e = 0.081819\\
    t = 1.284942\\
    $\eta^2 = 0.002542$\\
    $V^2 = 1.002542$\\
\end{center}

\indent Вычисляем три вспомогательные величины

\begin{center}
    $S \cdot \sin{\alpha}$, $S \cdot \cos{\alpha}$, $\Delta\alpha$
\end{center}

по формулам:

\begin{equation}
    S \cdot \sin{\alpha} = \frac{1}{f_2} \cdot \Delta\lambda \cdot \cos{\phi} \cdot \left(1 - f_3 \cdot (\Delta\lambda \cdot \sin{\phi})^2 + f_4 \cdot (\Delta\phi)^2 \right) 
\end{equation}

\begin{equation}
    S \cdot \cos{\alpha} = \frac{1}{f_1} \cdot \Delta\phi \cdot \cos{\left(\frac{\Delta\lambda}{2}\right)} \cdot \left(1 + f_5 \cdot (\Delta\lambda \cdot \cos{\phi})^2 + f_6 \cdot (\Delta\phi)^2 \right)
\end{equation}

\begin{equation}
    \Delta\alpha =  \Delta\lambda \cdot \sin{\phi} \cdot \left(1 + f_7 \cdot (\Delta\lambda \cdot \cos{\phi})^2 + f_8 \cdot (\Delta\phi)^2 \right)
\end{equation}

\indent Полученные значения:

\begin{center}
    $S \cdot \sin{\alpha} = -20553.618276^\circ$\\
    $S \cdot \cos{\alpha} = 1854.484618^\circ$\\
    $\Delta\alpha = -0.236752^\circ$\\
\end{center}

\indent По вычисленным значениям вычисляем окончательные значения исходных величин.\\
\indent Длина геодезической линии:

\begin{equation}
    S = \sqrt{\left(S \cdot \sin{\alpha} \right) + \left(S \cdot \cos{\alpha} \right)}
\end{equation}

\indent Величина азимута стандартно, т.е. как румб с учётом четверти:

\begin{equation}
    \alpha = \arctan\left (\left|\frac{S \cdot \sin{\alpha}}{S \cdot \cos{\alpha}} \right| \right)
\end{equation}

\begin{equation}
    \alpha_1 = \alpha - \frac{\Delta\alpha}{2}
\end{equation}

\begin{equation}
    \alpha_2 = \alpha + \frac{\Delta\alpha}{2} + \pi
\end{equation}

\indent Значение азимутов должно находиться в интервале

\begin{center}
    $0 \leqslant \alpha \leqslant 2 \pi$
\end{center}

\indent Полученные значения:
\begin{center}
    $S = 20637.110685 \textit{м}$\\
    $\alpha = 275.155647^\circ$\\
    $\alpha_1 = 275.274024^\circ$\\
    $\alpha_2 = 95.037272^\circ$\\
\end{center}

\newpage
\section{Прямая геодезическая задача на эллипсоиде}
В качестве исходных данных  имеем широту и долготу первой точки (\hyperlink{table_1}{Табл.1}), а также вычисленные значения прямого азимута и геодезической линии:

\begin{center}
    $P_1(\phi_1, \lambda_1)$, $\alpha_1$, S\\
\end{center}

\indent Широту и долготу второй точки, а также обратный азимут геодезической линии обозначим соответственно:

\begin{center}
    $P_2(\phi_2, \lambda_2)$, $\alpha_1$, $\alpha_2$\\
\end{center}

\indent Вычисляем приблизительные координаты точки $P_2$:

\begin{equation}
    \phi_2 = \phi_1 + \frac{S \cdot \cos{\alpha_1}}{M_1}
\end{equation}

\begin{equation}
    \lambda_2 = \lambda_1 + \frac{S \cdot \alpha_1}{N_1 \cdot \cos{\phi_1}}
\end{equation}

\indent Далее в ходе итерационного процесса вычисляются обратный азимут, широта и долгота второй точки по формулам:

\begin{equation}
    \Delta\lambda = \Delta\lambda \cdot \sin{\phi} \cdot \left (1 + f_7 \cdot (\Delta\lambda \cdot \cos{\phi})^2 + f_8 \cdot (\Delta\phi)^2 \right)
\end{equation}

\begin{equation}
    \alpha = \alpha_1 + \frac{\Delta\lambda}{2}
\end{equation}

\begin{equation}
    \alpha_2 = \alpha + \frac{\Delta\alpha}{2} + \pi
\end{equation}

\begin{equation}
    \lambda_1 + f_2 \cdot \frac{S \cdot \sin{\alpha}}{\cos{\phi}} \cdot \left(1 + f_3 \cdot (\Delta\lambda \cdot \sin{\phi})^2 - f_4 \cdot (\Delta\phi)^2 \right)
\end{equation}

\begin{equation}
    \phi_2 = \phi_1 + f_1 \cdot \frac{S \cdot \cos{\alpha}}{\cos{\frac{\Delta\lambda}{2}}} \cdot \left (1 - f_5 \cdot (\Delta\lambda \cdot \cos{\phi})^2 - f_6 \cdot (\Delta\phi)^2 \right)
\end{equation}

\indent Перед началом каждой итерации перевычисляются значения (\hyperlink{f_1}{1}) - (\hyperlink{f_8}{8}). Итерации следует продолжать до тех пор, пока значения текущей и предыдущей итерации не будут отличаться на пренебрегаемо малую величину, которая зависит от вида выполняемых работ. Результаты \hyperlink{table_2}{Табл.2} подведены за 3 итерационных цикла.

\begin{table}[h]
    \centering
    \hypertarget{table_2}{\caption{Журнал итераций}}
    \begin{tabular}{|c|c|c|c|}
    \hline
         $\textbf{Величина}$ & $\textbf{Итерация №1}$ & $\textbf{Итерация №2}$ & $\textbf{Итерация №}$  \\
         \hline
         $\Delta\alpha, ^\circ$ & -0.236664 & -0.236753=& -0.236752\\
         \hline
         $\alpha, ^\circ$ & 275.155692 & 275.155647 & 275.155647\\
         \hline
         $\alpha_2, ^\circ$ & 95.037360 & 95.037270 & 95.037271\\
         \hline
         $\lambda, ^\circ$ & 23.399999 & 23.399999 & 23.400000\\
         \hline
         $\phi, ^\circ$ & 52.116667 & 52.116667 & 52.116667\\
         \hline
    \end{tabular}
    \label{tab:my_label}
\end{table}

\indent Проверим правильность вычислений, выполнив обратный пересчёт по формулам параметрического способо описания эллипсоида.\\

\newpage
\section{Параметрический способ описания эллипсоида}
Эллипсоид WGS84 параметрически можно представить в виде формулы:

\begin{gather}
    \begin{pmatrix} x \\ y \\ z \end{pmatrix} = 
    \begin{pmatrix}
        N(\phi) \cdot \cos\phi \cdot \cos\lambda\\
        N(\phi) \cdot \cos\phi \cdot \sin\lambda\\
        N(\phi) (1 - e^2) \cdot \sin\phi
    \end{pmatrix}
\end{gather}

где первый вертикал $N(\phi)$ :

\begin{equation}
    N(\phi) = \frac{a}{\sqrt{1 - e^2 \sin^2{\phi}}}
\end{equation}

и эксцентриситет:

\begin{equation}
    \hypertarget{f_34}{e = \frac{\sqrt{a^2 - b^2}}{a} \Rightarrow e^2 = 2f - f^2}
\end{equation}

\indent Для точек над поверхностью эллипсоида необходимо в формуле (\hyperlink{f_34}{34}) добавить эллипосидальную высоту h (расстояние от точки до эллипсоида по нормали):

\begin{gather}
    \begin{pmatrix} x \\ y \\ z \end{pmatrix} = 
    \begin{pmatrix}
        (N+h) \cdot \cos\phi \cdot \cos\lambda\\
        (N+h) \cdot \cos\phi \cdot \sin\lambda\\
        (N(1 - e^2) +h) \cdot \sin\phi
    \end{pmatrix}
\end{gather}

\indent Для вычисления геодезических координат по пространственным прямоугольным можно испольховать формулы:

\begin{equation}
    \lambda = \arctan \left(\frac{Y}{X} \right)
\end{equation}

\begin{equation}
    \phi = \arctan \frac{Z + \varepsilon \cdot b \cdot \sin^3{q}}{p - e^2 \cdot a \cdot \cos^3{q}}
\end{equation}

где

\begin{equation}
   \varepsilon = \frac{e^2}{1 - e^2} 
\end{equation}

\begin{equation}
    b = a \cdot (1 - f)
\end{equation}

\begin{equation}
    p = \left (X^2 + Y^2 \right)^0,5
\end{equation}

\begin{equation}
    q = \arctan \left(\frac{Z \cdot a}{p \cdot b} \right)
\end{equation}

\indent Затем эллипсоидальная высота может быть найдена по формуле:

\begin{equation}
    h = \frac{p}{\cos{\phi}} - \nu
\end{equation}

\indent Полученные значения:

\begin{center}
     $X_1 = 3595062.049446 \textit{м}$ $ X_2 = 3601932.611383 \textit{м}$\\
     $Y_1 = 1578121.906722 \textit{м}$ $Y_2 = 1558695.427717 \textit{м}$ \\ 
     $Z_1 = 5009646.051168 \textit{м}$ $Z_2 = 5010785.024187 \textit{м}$\\
     
     $\varepsilon = 0.006739$\\
     $b = 6356752.3142458 \textit{м}$\\
     $p_1 = 3926186.431111^\circ$ $p_2 = 3924722.916758^\circ$\\
     $q_1 = 0.907688^\circ$ $q_2 = 0.907979^\circ$\\
     $\lambda_1 = 23.700000^\circ$ $\lambda_2 = 23.400000^\circ$\\
     $\phi_1 = 52.100000^\circ$ $\phi_2 = 52.116667^\circ$\\
     $S_\textit{геоц} = 20637.101720 \textit{м}$
\end{center}

\indent Вычислив разницу полученных значений, равную 0.008965 \textit{м} (8 мм). Из этого можно сделать вывод, что расстояние на сфере незначительно больше расстояния на эллипсоиде. Данная разница обусловлена особенностями фигур сферы (нормальный сфероид) и эллипсоида вращения (сплюснутый сфероид).

\end{document}
